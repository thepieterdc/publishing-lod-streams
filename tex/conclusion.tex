\section{Conclusion}
The goal of this paper was to investigate how existing technologies can be applied in the field of Linked Open Data to handle data streaming. We have performed a literature study and explained various techniques to define, version, distribute, and cache this data. Multiple existing and viable technologies have been described, and each section stressed that existing technologies could be combined to achieve better performance. A lot of research has been conducted at the protocol level, but further research at higher levels is recommended and may prove beneficial. Furthermore, it was noticed that existing pull-based algorithms, which perform better, have not yet been compared to a push-based approach. Finally, we propose that the technologies described in the different sections could be combined into a solution for distributing live open Linked Datasets with event streaming.\\

\noindent However, every dataset is different and therefore requires a different approach. For example, a dataset with parking spot availabilities needs an approach that prioritizes the speed of distribution of data changes, while an address registry dataset needs to prioritize access to this dataset at different times. We concluded in each section that various techniques exist to solve the subtask discussed in that section. However, no silver bullet exists (yet) that solves the complete problem. Therefore, we conclude that an exhaustive solution must use a combination of existing techniques and that future research should not consider the subtasks in isolation, but instead focus on the end-to-end combination of these subtasks. The benefit of reusing existing technologies, rather than establishing new standards, is the interoperability with existing systems.