\begin{abstract}
\noindent The Web in its current state cannot easily be interpreted by machines, as the majority of data resources and APIs are documented solely in natural language. This problem can be solved by augmenting resources on the web with semantic metadata; forming the so-called Semantic Web. An open problem is finding a generic approach to process such data in a streaming fashion and propagating updates to data reusers. An idiomatic solution to this problem should build on top of established Semantic Web standards, such as HTTP and RDF. This paper focuses on getting updates to the end-users as soon as possible, for which the applicability of existing technologies has been investigated. Our findings show that this subject can be further divided into four subtasks: (i) formatting, (ii) versioning, (iii) delivery, and (iv) caching. Although every subtask can be solved using existing technologies, none are sufficient to solve the larger problem. We conclude that the solution must lie in the combination of these existing technologies, and that future research should focus on end-to-end solutions -- and not only the smaller subtasks.\\

\noindent This paper is available at the following url:\\ \url{https://thepieterdc.github.io/publishing-lod-streams/}.
\end{abstract}