\section{Data streaming}
In this final section, different aspects are highlighted on how to maintain a stream of linked data. The first section presents a vision that compares the distributed storage of linked data with the current way of storing DNS records, through the use of intermediaries. The subsequent sections describe existing techniques specifically tailored towards linked data streaming, in the context of the earlier discussed formatting, versioning, delivery, and caching methods.

\subsection{Use of intermediaries}
Remark that we can observe an interesting correlation between chaining SPARQL endpoints and the distribution of datasets, versus the mechanics of \emph{Domain Name Servers (DNS)}. These servers start from a \emph{root} DNS server and delegate the request of the user through multiple intermediaries. The same idea can be applied when scaling linked data streams, or even linked datasets themselves. A root server could for example serve as an entry point for data publishers to push their data to, as well as for users to send their requests to. These requests can subsequently be routed via multiple intermediaries, depending on which part of the data is required. Consider, as an example, that we are interested in a stream of train delays for the Ghent-St. Pieters train station in Belgium. The root server could contain links to several intermediary servers that contain this data for each continent. The next level could divide this further into countries per continent, followed by a city-based level and eventually a station-based level. This way the distribution of the stream can even be scaled adaptively, depending on the need. This idea can be implemented using the topic-based strategy used in \cref{par:delivery-websub}.

\subsection{RDF Stream Processing}
The Internet of Things has caused a shift in the data landscape. Sensors publish data at high frequency into the cloud, so that it does not always make sense to use polling-based approaches. Instead, it is much more efficient to treat these values in a push-based (\cref{sec:delivery}) streaming fashion \cite{webofdatastreams}. To manage these data streams, Stream Processing Engines have seen the light. However, since the contents of the stream can be in all shapes and sizes, a Web of Data-counterpart was needed. This has lead to the creation of RDF Stream Processing techniques, which allow to process RDF-based data streams using various operations such as filtering and aggregating.

\subsubsection{TripleWave}\label{sssec:streaming-rdfsp-triplewave}
TripleWave is an open-source framework that allows the creation and the publishing of RDF streams. In their requirements section, Mauri et al. \cite{mauri2016triplewave} list seven requirements to which the framework must comply. The most interesting requirements are \texttt{R1} and \texttt{R2}, which respectively state that ``TripleWave may use streams available on the Web as input'' and that ``TripleWave shall be able to process existing time-aware (RDF) datasets'' (which could either be formatted as a stream or not). These requirements imply that the framework has conversion mechanisms to transform this data. The framework uses CSV and JSON connectors to convert existing web streams, as well as R2RML (\cref{formatting-decoupling}) for the generation of RDF streams.\\

\noindent For consumption of the streams, the framework provides both a push-based and pull approach, using WebSockets, end users can subscribe to a TripleWave endpoint and get updates pushed to them. Alternatively, streams can also be published for the user to pull them.

\subsubsection{Web Stream Processors}
Dell'Aglio et al. \cite{webofdatastreams} propose an infrastructure that is capable of handling TripleWave (\cref{sssec:streaming-rdfsp-triplewave}) as input, and impose seven requirements for a Web of Data Streams, called \emph{Web Stream Processors (WeSP)}. These requirements are inspired by previous work from Stonebraker et al. \cite{10.1145/1107499.1107504}.

\begin{enumerate}
    \item \textbf{Keep the data moving:} The first requirement is that \emph{WeSP must prioritize active paradigms for data stream exchange, where the data supplier can push the stream content to the actors interested in it}. This means that instead of using polling-based approaches, data publishers must push updates to subscribers. This has been discussed in Data delivery (\cref{sec:delivery}) and matches the findings for low latency updates by Van de Vyvere et al. \cite{van2020comparing,10.1145/3184558.3191650}, since streams with high frequency correspond to real-time data flows.

    \item \textbf{Stored and streamed data:} As a second requirement, \emph{WeSP must enable the combination of streaming and stored data}. Stored data in this context refers to data which has been formatted using the open data standards described in Data formatting (\cref{sec:formatting}), such as RDF. Additionally, this data can be exposed and queried via existing SPARQL-endpoints. To allow both streaming and storing the data, one of the RDF archive storage-techniques (\cref{versioning-rdfarchives-storage}) can be used. Specifically for streaming-based data, the delta-related query atoms are the predominant use case and therefore a change-based approach should be preferred.

    \item \textbf{High availability, distribution and scalability:} Next, Dell'Aglio et al. \cite{webofdatastreams} require that \emph{WeSP must enable the possibility to build reliable, distributed and scalable streaming applications}. Obtaining high availability is possible using either one of the provided Data versioning (\cref{sec:versioning}) approaches. The Git-based approaches in particular allow a dataset to be distributed efficiently across multiple devices and therefore improve the scalability and reliability. Depending on the frequency of the updates, the reliability can be further improved through caching.

    \item \textbf{Operations on the stream content:} A subsequent requirement is that \emph{WeSP must guarantee a wide range of operations over the streams}. As mentioned before, RDF Stream Processing requires that the stream can be filtered or aggregated and that the result is another stream, which can be observed. This can be implemented via SPARQL queries that receive streamed data as their input, apply the desired operation, and return a new output stream of the results. Multiple of these queries can be chained together, to obtain the desired output depending on the use case.

    \item \textbf{Accessible information about the stream.} In order to improve the accessibility of the data, \emph{WeSP must support the publication of stream descriptions}. These stream descriptions can contain metadata, such as the expected update frequency and the expected size of the updates. This information can serve multiple purposes. If a Git-based approach would be used to store the data, the meta-information can indicate the parent version, which can subsequently be interpreted to update a local copy by resolving the deltas. Additionally, if applicable, \texttt{Expiry} timestamps can be added to benefit from Data caching (\cref{sec:caching}).
    
    \item \textbf{Stream variety support:} The penultimate requirement is that \emph{WeSP should support the exchange of a wide variety of streams}. Since the web is of decentralized nature, we cannot define a single model and format to which all streams must adhere. Instead, every stream can have its own format, depending on the content type and publishing frequency. Various formats discussed in \cref{sec:formatting} can be used.
    
    \item \textbf{Reuse of technologies and standards:} Finally, \emph{WeSP should exploit as much as possible existing protocols and standards}. A manifold of existing technologies and infrastructures already exist. In the context of formatting, versioning, delivery, and caching, this paper has provided sufficient examples. These existing technologies should be reused whenever possible to encourage interoperability. While starting from scratch can often give good results on a local scale (e.g. devising a format that is specifically created for the particular use case of the data), but prevents the technology from being globally adopted.
\end{enumerate}